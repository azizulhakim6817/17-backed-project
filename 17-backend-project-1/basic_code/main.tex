###<< Epi-1 >>><<<<< Mongoose introduction >>>>>>>>>>>>>>>>>>>>>>>>>>>>>>>>>>>>>>>>>>>>>>>>>>>>>>>>>>>>>>>>>>>>>>>>>
@@@<< part-1 >>><<<<<< Mongoose ??????????????????????????

1. What is Mongoose?
   Mongoose an elegant MongoDB object modeling for Node.js.

2. Where is used?
   1. Provides Schema-base data modeling solution.
   2. Type casting
   3. Validation
   4. Query building
   5. Business logic

###<< Epi-2 >>><<<<< Mongoose installation >>>>>>>>>>>>>>>>>>>>>>>>>>>>>>>>>>>>>>>>>>>>>>>>>>>>>>>>>>>>>>>>>>>>>>>>>
@@@<< part-1 >>><<<<<< Mongoose ??????????????????????????

1. How to start?
   1. install Mongoose
   2. Create Database connection
   3. Create fist Mongoose Model 
   4. Work with mongoose model

   @@@ installation packages :::::::::::::::::
   "dependencies": {
      "body-parser": "^1.20.2",
      "cookie-parser": "^1.4.6",
      "cors": "^2.8.5",
      "dotenv": "^16.4.5",
      "express": "^4.19.2",
      "express-mongo-sanitize": "^2.2.0",
      "express-rate-limit": "^7.4.0",
      "helmet": "^7.1.0",
      "hpp": "^0.2.3",
      "jsonwebtoken": "^9.0.2",
      "mongoose": "^8.6.1",
      "multer": "^1.4.5-lts.1",
      "mysql": "^2.18.1",
      "nodemon": "^3.1.4",
      "xss-clean": "^0.1.4"
    }

###<< Epi-3 >>><<<<< Mongoose Database connection >>>>>>>>>>>>>>>>>>>>>>>>>>>>>>>>>>>>>>>>>>>>>>>>>>>>>>>>>>>>>>>>
@@@<< part-1 >>><<<<<< Database connection ??????????????????????????

   1. uri maining? =>  Uniform Resource Identifier 

//MongoDB Database connection..............................
/* let URI = "mongodb://127.0.0.1:27017/studentID";
let OPTION = { user: " ", pass: " " };
mongoose
  .connect(URI, ({autoIdex : true}))
  .then(() => {
    console.log("MongoDB is connnected");
  })
  .catch((err) => {
    console.log("Database Error", err);
  }); */

//...................................................................
/* let MONGODB_CONNECTION = "mongodb://127.0.0.1:27017/studentID";
mongoose
  .connect(MONGODB_CONNECTION)
  .then(() => {
    console.log("Database Connected");
  })
  .catch((err) => {
    console.log("Database Error", err);
  }); */

//........................................................
/* let URI = "mongodb://127.0.0.1:27017/studentID";
mongoose
  .connect(URI, {
    useNewUrlParser: true,
    useUnifiedTopology: true,
  })
  .then(() => {
    console.log("Database connection successful");
  })
  .catch((err) => {
    console.error("Database connection error:", err);
  }); */


###<< Epi-3 >>><<<<< Mongoose Database connection >>>>>>>>>>>>>>>>>>>>>>>>>>>>>>>>>>>>>>>>>>>>>>>>>>>>>>>>>>>>>>>>
@@@<< part-1 >>><<<<<< Database connection ??????????????????????????
//MongoDB Database connection.......
let URI = "mongodb://127.0.0.1:27017/studentID";
let OPTION = { user: " ", pass: " " };

mongoose
  .connect(URI)
  .then(() => {
    console.log("db is connected!");
  })
  .catch((error) => {
    console.log("db is not connected!");
    console.log(error.message);
    process.exit(1);
  });

###<< Epi-4 >>><<<<< Schema => Mongoose connnected >>>>>>>>>>>>>>>>>>>>>>>>>>>>>>>>>>>>>>>>>>>>>>>>>>>>>>>>>>>>>>>>
@@@<< part-1 >>><<<<<< Schema ??????????????????????????
const mongoose = require("mongoose");

const DataSchema = mongoose.model({
  Name: String,
  Roll: String,
  Class: String,
  Remarks: String,
});

const studentsModel = mongoose.model("Studetnts", DataSchema);

module.exports = studentsModel;

###<< Epi-5 >>><<<<<>> create data  >>>>>>>>>>>>>>>>>>>>>>>>>>>>>>>>>>>>>>>>>>>>>>>>>>>>>>
@@@<< part-1 >>><<<<<< create data using Model ??????????????
1. packages :::::::::::::::::::::::::::::::::::::
   @@@ const bodyParser = require("body-parser");
   @@@ app.use(bodyParser.json());

2. const studentsModel = require("../models/user.js");

// create data
exports.InsertStudent = async (req, res) => {
  try {
    const studentData = req.body; // Renamed to better reflect the content
    const newStudent = await studentsModel.create(studentData);

    res.status(201).json({
      status: "success",
      data: newStudent,
    });
  } catch (error) {
    res.status(400).json({
      status: "fail",
      error: error.message || "An error occurred while inserting the student",
    });
  }
};

###<< Epi-6 >>><<<<<>> Read Data  >>>>>>>>>>>>>>>>>>>>>>>>>>>>>>>>>>>>>>>>>>>>>>>>>>>>>>
exports.readStudent = async (req, res) => {
  try {
    let Query = {};
    let Projection = "Name Roll Class Remarks ";
    let data = await studentsModel.find(Query, Projection);

    res.status(201).json({ status: "Success", data: data });
  } catch (error) {
    res.status(401).json({ status: "fail", data: error });
  }
};

###<< Epi-7 >>><<<<<>>  Update Data  >>>>>>>>>>>>>>>>>>>>>>>>>>>>>>>>>>>>>>>>>>>>>>>>>>>>>>

exports.updateStudent = async (req, res) => {
  let id = req.params.id;
  let query = { _id: id };
  let updateData = req.body;

  try {
    let data = await studentsModel.updateOne(query, updateData);
    res.status(201).json({ status: "Success", data: data });
  } catch (error) {
    res.status(401).json({ status: "fail", data: error });
  }
};

###<< Epi-8 >>><<<<<>> delete data  >>>>>>>>>>>>>>>>>>>>>>>>>>>>>>>>>>>>>>>>>>>>>>>>>>>>>>
//delete...........................................
exports.deleteStudent = async (req, res) => {
  let id = req.params.id;
  let query = { _id: id };

  try {
    let data = await studentsModel.deleteOne(query);
    res.status(201).json({ status: "Success", data: data });
  } catch (error) {
    res.status(401).json({ status: "fail", data: error });
  }
};

###<< Epi-9 >>><<<<<>> mongoose Schema types   >>>>>>>>>>>>>>>>>>>>>>>>>>>>>>>>>>>>>>>>>>>>>>>>>>>>>>
    1. String
    2. Number
    3. Data
    4. Buffer
    5. Boolean
    6. Mixed
    7. ObjectId
    8. Array
    9. Decimal28
    10. Map

  const mongoose = require("mongoose");
  const DataSchema = mongoose.Schema({
      Name: {type: String},
      Roll: {type: Number },
      Class: {type: String},
      Remarkes: {type: String},
      Adult: {type: boolean},
      comment: [],
      details: {},
      dob: Date.now,
    });
    
    const ModelSchema = mongoose.model("users", DataSchema);
    
    export default ModelSchema;

###<< Epi-10 >>><<<<<>> mongoose default Value/Version Key >>>>>>>>>>>>>>>>>>>>>>>>>>>>>>>>>>>>>>>>>>>>>>>>>>>>
1. date: {type: Date, default: Date.new}
2. mongoose default key =  _id
3. {versionKey: false} 
4. Remarkes: { type: String, default: "No Remarks" },


###<< Epi-11 >>><<<<<>> mongoose Validation-1 >>>>>>>>>>>>>>>>>>>>>>>>>>>>>>>>>>>>>>>>>>>>>>>>>>>>>>>>>
1. Type casting Validation = class: {type: String}

1. Required Validation = Name: {type: String, unique: true}

3. Min-Max Number Validation = Roll: {type: Number, min: 6, max: 12}

3. Roll: {
        type: Number,
        min: [6, "Minimum Roll 6 & Maximum Roll 12, But got {VALUE}"];
        max: [12, "Minimum Roll 6 & Maximum Roll 12, But got {VALUE}"]
}

4. Enumerated type Validation : 
   Class: {type: enum, {value: ["one", "two", "three", "four, "six"], message: "{VALUE} is not supported"}}

5. Custom Validation :
   Mobile: {
     type: String,
     validate: {
     validato: funtion(value){
        if(value.lenght!==11){
          returnt false;
        }else{
          return true;
        }
      }
      message: "11 Digit Phone Number Required";
     }
   }

6. Regex Validation:
   Mobile: {
    type: String,
    validate: {
      funtion(value) {
        return /^(?:\+?88|0088)?01[15-9]\d{8}$/test.(value);
      },
      message: porps =>"${porps.value} is not a valid phone Numbeer" ;
    }
   
   }

###<< Epi-12 >>><<<<<>> mongoose Validation-2 >>>>>>>>>>>>>>>>>>>>>>>>>>>>>>>>>>>>>>>>>>>>>>>>>>>>>>>>>
1. const mongoose = require("mongoose");

const stuenetSchema = mongoose.Schema(
  {
    Name: { type: String },
    Roll: { type: String },
    Class: { type: String },
    },
    { versionKey: false }
    );
    const studentsModel = mongoose.model("userStudentsID", stuenetSchema);
    module.exports = studentsModel;
    
###<< Epi-13 >>><<<<<>> mongoose Validation-3 >>>>>>>>>>>>>>>>>>>>>>>>>>>>>>>>>>>>>>>>>>>>>>>>>>>>>>>>>
1. Roll => 6,7,8,9,10,11,12 .................
Roll: {
  type: Number,
  min: [6, "Minimum Roll 6 & Maximum Roll 12, But got {VALUE}"];
  max: [12, "Minimum Roll 6 & Maximum Roll 12, But got {VALUE}"]
  }

###<< Epi-14 >>><<<<<>> mongoose Validate/validato-4 >>>>>>>>>>>>>>>>>>>>>>>>>>>>>>>>>>>>>>>>>>>>>>>>>>>>>>>>>
const mongoose = require("mongoose");

const stuenetSchema = mongoose.Schema(
  {
    Name: { type: String },
    Roll: { type: Number },
    Class: { type: String },
    Mobile: {
      type: String,
      validate: {
        validator: function (value) {
          if (value.lenght === 11) {
            return true;
          } else {
            return false;
          }
        },
        message: "11 Digite Mobile Number Required",
      },
    },
  },
  { versionKey: false }
);
const studentsModel = mongoose.model("userStudentsID", stuenetSchema);
module.exports = studentsModel;

###<< Epi-15 >>><<<<<>> mongoose Validation Regex-5 >>>>>>>>>>>>>>>>>>>>>>>>>>>>>>>>>>>>>>>>>>>>>>>>>>>>>>>>>
      1. Regex Bangladesh monbil regex code =>   https://stackoverflow.com/questions/30658946/validate-bangladeshi-phone-number-with-optional-88-or-01-preceeding-11-digits

      2. /(^(\+88|0088)?(01){1}[3456789]{1}(\d){8})$/
  
      const mongoose = require("mongoose");

      const stuenetSchema = mongoose.Schema(
        {
          Name: { type: String },
          Roll: { type: Number },
          Class: { type: String },
          Mobile: {
            type: String,
            validate: {
              validator: function (value) {
                return /(^(\+88|0088)?(01){1}[3456789]{1}(\d){8})$/.test(value);
              },
              message: "Invalid Bangladesh Mobile Number",
            },
          },
        },
        { versionKey: false }
      );
      const studentsModel = mongoose.model("userStudentsID", stuenetSchema);
      module.exports = studentsModel;

###<< Epi-16 >>><<<<<>> Documentation Using Postman-1 >>>>>>>>>>>>>>>>>>>>>>>>>>>>>>>>>>>>>>>>>>>>>>>>>>>>>>>>>
1. thred party library => postman / 
2. download
3. Open accout create / sign up 

###<< Epi-17 >>><<<<<>> Documentation Using Postman-2 >>>>>>>>>>>>>>>>>>>>>>>>>>>>>>>>>>>>>>>>>>>>>>>>>>>>>>>>>
1. workspaces -> create => project-1
2. collection -> ceate => demo 
3. create folder => userID
3. add request  => InsertStudent-method-post
3. add request  => read-method-get

###<< Epi-18 >>><<<<<>> Documentation Using Postman-3 >>>>>>>>>>>>>>>>>>>>>>>>>>>>>>>>>>>>>>>>>>>>>>>>>>>>>>>>>
1. project..........
<----Variable-------------------internal value------------------current value----------------------->
<-----BASE_URL---------------http://localhost:9000/-------------http://localhost:9000/

2. Authentication ?????????????????????????????????
<----API Key---------------------------------------->
1. Key =>
2. Value => 
2. Add to => headers / Query params

<----Beara token---------------------------------------->
1. token

<----Basic Auth---------------------------------------->
1. username
2. password

<----Windows NTLM authentication---------------------------------------->
1. username
2. password
3. Domain
4. workstation

###<< Epi-19 >>><<<<<>> Documentation Using Postman-4 >>>>>>>>>>>>>>>>>>>>>>>>>>>>>>>>>>>>>>>>>>>>>>>>>>>>>>>>
1. project..........
<----Variable-------------------internal value------------------current value----------------------->
<-----BASE_URL---------------http://localhost:9000/-------------http://localhost:9000/

2. postman => {{BASE_URL}}/InsertStudent

//@@@@ body..............................................
1. mongoose object create => Schema => body

2. Postman => body => row => json => { "name" : "Azizul" "Roll" : "144"} 

//@@@ headers ..............................
1. By default call headers => 9 hidden - by default postman set 
2. 1. cache-control 2. postman-control 3. host 4. user-agent 5. accept-uncoding etc....

//@@@ Authentication..........................................
1. no-auth
2. set all times => inherit auth parent

//@@@ Query Params.........................................
1. name-================== value =========description

###<< Epi-20 >>><<<<<>> c + u + r + postman documnet-5 >>>>>>>>>>>>>>>>>>>>>>>>>>>>>>>>>>>>>>>>>>>>>>>>>>>>>>>>
const studentsModel = require("../models/StuentsModel.js");

// create data........................
exports.InsertStudent = async (req, res) => {
  try {
    const studentData = req.body;
    const newStudent = await studentsModel.create(studentData);

    res.status(201).json({ status: "success", data: newStudent });
  } catch (error) {
    res.status(400).json({
      status: "fail",
      error: error.message || "An error occurred while inserting the student",
    });
  }
};

//read data ................
exports.readStudent = async (req, res) => {
  try {
    let Query = {};
    //let Projection = "Name Roll Class Remarks ";
    let data = await studentsModel.find(Query);

    res.status(201).json({ status: "Success", data: data });
  } catch (error) {
    res.status(401).json({ status: "fail", data: error });
  }
};

//update data ................
exports.updateStudent = async (req, res) => {
  let id = req.params.id;
  let query = { _id: id };
  let updateData = req.body;

  try {
    let data = await studentsModel.updateOne(query, updateData);
    res.status(201).json({ status: "Success", data: data });
  } catch (error) {
    res.status(401).json({ status: "fail", data: error });
  }
};

###<< Epi-21 >>><<<<< collection export / import / download >> Using Postman-6 >>>>>>>>>>>>>>>>>>>>>>>>>>>
 1. psotname ..............
    => collection export / import / download

  //delete...........................................
  exports.deleteStudent = async (req, res) => {
  let id = req.params.id;
  let query = { _id: id };

  try {
    let data = await studentsModel.deleteOne(query);
    res.status(201).json({ status: "Success", data: data });
  } catch (error) {
    res.status(401).json({ status: "fail", data: error });
  }
};

###<< Epi-22 >>><<<<< Documentation Using Postman-7 >>>>>>>>>>>>>>>>>>>>>>>>>>>>>>>>>>>>>>>>>>>>>>>>>>>>>>
1. Folder click
2. view complete Documentation 
3. Publish 
4. Publish URL => https://documenter.getpostman.com/view/36176527/2sAXqmAQyZ

###<< Epi-23 >>><<<<< Documentation Using Postman-7 >>>>>>>>>>>>>>>>>>>>>>>>>>>>>>>>>>>>>>>>>>>>>>>>>>>>>>
1. language seleted 
2. URL 
3. all documenter

###<< Epi-24 >>><<<<< JWT Token Authentication-1 >>>>>>>>>>>>>>>>>>>>>>>>>>>>>>>>>>>>>>>>>>>>>>>>>>>>>>>>
#### authentication & authrization ++++++++++++++++
1. API Key = threed party authentication
2. Bearer Token
3. Basic Auth
4. Digest Auth
5. Auth 1.0
6. Auth 2.0
7. Hawk Authentication
8. AWS Singnature
9. NTLM authentication = Windows authentication

###<< Epi-25 >>><<<<< Encorded-Decoded-create.middleware-applly.routes >>>>>>>>>>>>>>>>>>>>>>>>>>>>>>>>>>>>>
@@@ jwt = jsonwebtoken => jwt 

@@@ Encoded---------------Ternsfer------------Decoded--------------->
    oifsdalkj23-0djs -----------------------json form---------------------->    

@@@ auth => jwt 
1. How to Encoded json and create token.
2. How to Decoded (json) and retrieve json 
3. How to create auth middleware in express project
4. How to applly auth middleware in Routes

###<< Epi-26 >>><<<<< JWT Token Authentication-2 >>>>>>>>>>>>>>>>>>>>>>>>>>>>>>>>>>>>>>>>>>>>>>>>>>>>>>>>>>>>>

1. Postname => Get =>  {{BASE_URL}}/CreateEncodedToken
   token_key  => eyJhbGciOiJIUzI1NiIsInR5cCI6IkpXVCJ9.eyJleHAiOjE3MjYxNDIzNDUsImRhdGEiOnsiTmFtZ
   Received token_key => Copy

2. postname => Get => {{BASE_URL}}/DecordToken
   headers => token_key = Copy - past 

const jwt = require("jsonwebtoken");
"status": "success token",
    "data": {
        "exp": 1726141688,
        "data": {
            "Name": "Aziz",
            "Roll": "3029",
            "Class": "programs"
        },
        "iat": 1726138088
    }

................................ VS-code .......................................................
exports.CreateEncodedToken = (req, res) => {
  let Playload = {
    exp: Math.floor(Date.now() / 1000) + 60 * 60,
    data: { Name: "Aziz", Roll: "3029", Class: "programs" },
  };

  let Token = jwt.sign(Playload, "ScreatKey123");

  res.send(Token);
};

exports.DecordToken = (req, res) => {
  let token = req.headers["token_key"];
  jwt.verify(token, "ScreatKey123", function (err, decoded) {
    if (err) {
      res.stastus(401).json({ status: "invalid token", data: err });
    } else {
      res.status(200).json({ status: "success token", data: decoded });
    }
  });
};

###<< Epi-27 >>><<<<< JWT Token Authentication-2 >>>>>>>>>>>>>>>>>>>>>>>>>>>>>>>>>>>>>>>>>>>>>>>>>>>>>>

###<< Epi-28 >>><<<<< JWT Token Authentication-2 >>>>>>>>>>>>>>>>>>>>>>>>>>>>>>>>>>>>>>>>>>>>>>>>>>>>>>

1. <----request---------middleware--------------response------------>
2. tokenIssue.Contorller.js => 
2. create  middleware => use Routes(api.js) => 

// @@@ Encoded....token create ................................
const jwt = require("jsonwebtoken");

exports.TokenIssue = (req, res) => {
  let Playload = {
    exp: Math.floor(Date.now() / 1000) + 60 * 60,
    data: { Name: "Aziz", Roll: "3029", Class: "programs" },
  };

  let Token = jwt.sign(Playload, "ScreatKey123");

  res.send(Token);
};

// @@@ Decoded token verify ............................

const jwt = require("jsonwebtoken");

module.exports = (req, res, next) => {
  let token = req.headers["token_key"];
  jwt.verify(token, "ScreatKey123", function (err, decoded) {
    if (err) {
      res.stastus(401).json({ status: "invalid token", data: err });
    } else {
      next();
    }
  });
};

###<< Epi-29 >>><<<<< Project introduction-1 >>>>>>>>>>>>>>>>>>>>>>>>>>>>>>>>>>>>>>>>>>>>>>>>>>>>>>>>>>>>
@@@ Project Covers: ...........................
1. Rest API introduction
2. json Best practice
3. request response model
4. Rest API Bssic best practice
5. Rest API Security practice
6. Express.js Rest API Fundametal
7. MongoDB Core Fundametal 
8. mongoose
9. Documentation
10. jwt Auth 

@@@ project features :..........................
1. user Registration
2. user Loing 
3. user Auth By JWT Token
4. user Profile Read
5. user Profile update
6. user to-do list create
7. user to-do list read
8. user to-do list update
9. user to-do list delete 
10. user to-do list Complete/Cancel Mark 
11. user to-do list complete/Cancel Filter
12. user to-do list Date Filter

###<< Epi-30 >>><<<<< create new project-1 >>>>>>>>>>>>>>>>>>>>>>>>>>>>>>>>>>>>>>>>>>>>>>>>>>>>>>>>>>>>